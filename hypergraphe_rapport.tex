%
% Exmple LateX de Julien DAVID, 30-01-07 
%
\documentclass[a4paper,11pt]{article}
\usepackage[utf8]{inputenc}
\usepackage[francais]{babel}
\usepackage[T1]{fontenc}      % un second package
\usepackage{fancyhdr}
\usepackage{amsmath}




\newcommand{\sommaire}{\shorttoc{Sommaire}{1}}

\author{Mehdi Naima\\ Université Paris Nord}
\title{Hypergraphes}


\begin{document}
\begin{abstract}

\end{abstract}
\maketitle








\newpage

\tableofcontents


\pagestyle{fancy}
\renewcommand\headrulewidth{0.5pt}
\fancyhead[L]{Génération aléatoire des hypergraphes}
\newpage
\listoffigures
\listoftables
\newpage

\newfont{\joli}{cmfi10 scaled \magstep2}
 


%**************************************
\section{Rappel}
%**************************************

%--------------------------------------
\subsection{Hypergraphe}
%--------------------------------------
Les hypergraphes sont des objets mathématiques généralisant la notion de graphe. \\
Un hypergrapge H est un couple (V,E)

\begin{itemize}
\item
element 1
\item
element 2\footnote{note de bas de page}
\end{itemize}



\end{document}





